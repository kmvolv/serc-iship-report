\cleardoublepage
%\pagebreak
\sectionfont{\fontsize{18}{15}\selectfont}
\section*{Abstract}
\begin{FlushLeft}
\textbf{Software Engineering Research Center (SERC)} has a eminent faculty with vast teaching and research experience in and outside India. SERC has close collaboration with industry providing software services to large organizations, R\&D labs of various organizations and other academic institutes (India and abroad). Organizations that SERC is working are diverse, stratified across banking and finance, government, equipment manufacture, ISVs building products for software industry, etc.\\[0.5in]
\end{FlushLeft} 

\begin{FlushLeft}
\textbf{Algodynamics} : \emph{A new approach to learning algorithms}\\[0.1in]
Algorithms are the cornerstone of introductory Computer Science courses for engineering students. This is one of the most directly relevant courses for students who join industry after an undergraduate course. However, it is felt that the traditional approach to teaching algorithms in engineering colleges have a few challenges that have still not been addressed.\\[0.1in] For example, most algorithm textbooks and classroom are very prescriptive – they describe the algorithm in a particular way using pseudo-code or program and expect the students to understand them as-is, without helping them understand why. Similarly, tinkering and experimenting are critical for learning for engineering students, but these approaches don’t get applied for algorithm teaching.\\[0.1in]

Algodynamics applies system dynamics approach to describe and teach algorithms and aims to address some of these challenges.\\[0.1in]

For experiencing the pedagogy, visit the \href{https://algodynamics.io/labs.html}{Labs} or \href{https://algodynamics.io/experiments.html}{Experiments} page.\\[0.1in]
\end{FlushLeft}