\cleardoublepage

\section{Introduction}
\\[1in]
\begin{FlushLeft}

\subsection{Learning Objectives}
\begin{itemize}
    \item To get familiarized with the language of Elm and in turn getting a good idea of functional programming.
    \item To get an idea of how industry-standard code should be made, in order to make it easy to read and modify.
    \item To make dynamic code which can automatically adjust its parameters depending on the users input.
    \item Using inference obtained from existing packages to make our own visualization package with which a given data structure can be visualized. 
\end{itemize}
\\[0.5in]
\subsection{About Internship}
As an intern with SERC, I was given the responsibility of contributing to a visualization package which will be utilized in the algodynamics page. This would help in an easier implementation with the added possibility to include more features or handle different events.\\ [0.1in]
During my internship with the research center, I was able to grasp a detailed understanding of the functional programming language, Elm. With the help of various learning resources like \href{https://elmprogramming.com/}{Beginning \|> Elm} and \href{https://guide.elm-lang.org/}{An Introduction to Elm}.\\[0.1in]
By utilizing the language of Elm, I was able to realize the importance of functional programming and how it can be a much better option than compared to the extremely popular language of JavaScript as it has features such as Static Typing and Immutable Data.\\[0.1in]

I was able to get a good understanding on how to implement existing packages and correctly format Elm code through a few projects which will be explained about in detail in the later sections.\\[0.1in]
\end{FlushLeft}
\cleardoublepage
\begin{FlushLeft}
\subsection{Installation and Integration}
A detailed description on the installation of elm can be found \href{https://guide.elm-lang.org/install/elm.html}{here}\\[0.1in]
To check whether your installation was successful, you can enter \texttt{elm --version} in your terminal. If elm was installed properly, you should find the version number being displayed in the terminal.\\[0.1in]
Common commands used :\\[0.1in]
\begin{itemize}
    \item \texttt{elm init} : Initializes an elm project with an \texttt{elm.json} file and \texttt{src} directory. \texttt{elm.json} is used to describe the dependencies and \texttt{src} is used to store Elm files.
    \item \texttt{elm reactor} : Builds elm project and starts a server at \href{http://localhost:8000/}{http://localhost:8000/}
    \item \texttt{elm make} : Compiles Elm code to HTML or JavaScript 
    \item \texttt{elm install} : Used to install packages from \href{https://package.elm-lang.org/}{package.elm-lang.org}. Installed packages are added as dependencies in the \texttt{elm.json} file.
\end{itemize}
\end{FlushLeft}